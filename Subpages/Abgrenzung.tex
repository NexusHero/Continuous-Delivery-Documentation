\section{Abgrenzung zu anderen Arbeiten}
Diese Arbeit stützt sich auf die Inhalte der Bücher ''Continuous Delivery''\cite{Wolff.2016} und ''Product-Focused Software Process''\cite{Jedlitschka.2014}. Diese beinhalten sowohl die benötigten Grundlagen als auch fachlich tiefer greifende Aspekte. Die hier vorgestellten Vorgehensweisen werden verwendet, um Umsetzungsstrategien für Unternehmen herauszuarbeiten. \\ Bei der Umsetzung dieser Strategien entstehen allerdings Probleme, die wiederum dazu führen, dass die Umsetzung von Continuous Delivery erschweren kann. Hierfür ist eine Problemanalyse erforderlich. Für diese Analyse stützt sich diese Arbeit auf die Paper Laukkanen et al\cite{Laukkanen.2017}, Shahin et al\cite{Shahin.2017}. Die beiden Paper beleuchten eine Menge von Case-Studies und Literaturreviews, welche die vorhin erwähnten Probleme und auch deren Ursachen aufzeigen. Diese Arbeiten haben alle eine Gemeinsamkeit: Die Frage nach der Adaption von Continuous Delivery wird recht im Allgemeinen beantwortet und ist somit weiter Weg für den praktischen Einsatz in Unternehmen. Auswirkungen in Form von wie zum Beispiel Mitarbeiterschulungen, Kostenersparnis oder Prozessanpassungen wurden bisher nicht beleuchtet, welche allerdings eine wichtige Rolle einnehmen. Diese Fragen soll die vorliegende Arbeit von den anderen Arbeiten abgrenzen.  