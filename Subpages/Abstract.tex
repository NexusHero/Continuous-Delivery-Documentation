% for Computer Society papers, we must declare the abstract and index terms
% PRIOR to the title within the \IEEEtitleabstractindextext IEEEtran
% command as these need to go into the title area created by \maketitle.
% As a general rule, do not put math, special symbols or citations
% in the abstract or keywords.
\IEEEtitleabstractindextext{%
	\begin{abstract}
	Continuous Integration kommt heutzutage in den meisten Unternehmen zum Einsatz. Nach dem Commit des geänderten Codes, wird dieser automatisch gebuildet und getestet, wodurch der Entwickler sehr schnell Feedback zu Fehlern erhält. Dadurch hat man bereits einen stabilen Prozess, um nachhaltig qualitativ hochwertige Software in Produktion zu bringen. An dieser Stelle setzt nun Continuous Delivery an. Die Prinzipien, die bei Continuous Delivery verfolgt werden klingen im ersten Moment vielversprechend, denn es ermöglicht das Ausrollen von Software mit einer wesentlich höheren Geschwindigkeit sowie Zuverlässigkeit. Grundlage dafür ist die Continuous-Delivery-Pipeline, die das in Produktion bringen der Software weitgehend automatisiert und so einen reproduzierbaren, risikoarmen Prozess für die Bereitstellung neuer Releases darstellt [1]. 
	
	Obwohl Continuous Delivery lediglich eine Sammlung von Techniken, Prozessen und Werkzeugen ist, kann sich das Aufsetzen dieser Disziplin sehr schwierig gestalten. In diesem wissenschaftlichen Artikel wird die Frage behandelt, was ein Unternehmen aufwenden muss, um diesen Schritt zu gehen und welche Auswirkungen diese Entscheidung auf das Unternehmen hat. 
	Dabei wird gezeigt, welche Rolle DevOps bei diesem Schritt spielt, welche Komponenten ein Unternehmen dabei betrachten muss und welche Barrieren es gibt. Abschließend wird gezeigt, welche Änderungen in der Unternehmenskultur man dabei zu erwarten hat.
	
	\end{abstract}
	
	% Note that keywords are not normally used for peerreview papers.
	\begin{IEEEkeywords}
		Computer Society, IEEE, IEEEtran, journal, \LaTeX, paper, template.
\end{IEEEkeywords}}