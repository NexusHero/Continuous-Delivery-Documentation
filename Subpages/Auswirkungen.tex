\section{Auswirkungen auf das Unternehmen}
Dadurch, dass eine DevOps-Kultur anders ist, ergibt dies eine andere Mentalität. 
DevOps ermöglicht es aussagekräftige ROI-Metriken zu liefern. Ob die Veränderung Auswirkungen hat, sagen einem die Zahlen. Wichtig ist es somit, entsprechend zu messen.
Unternehmen müssen wissen, wie effizient oder ineffizient bestehende Softwareentwicklungsprozesse sind, bevor sie sich zu verändern beginnen.
Verstehen Sie auch, welche Ergebnisse Sie erwarten und planen Sie entsprechend. Basierend auf unserer Erfahrung sollte die Bereitstellung acht- bis zehnmal schneller sein, mit einer Qualitätssteigerung von 50 Prozent (Messung der Reduzierung von Vorfällen und Vorfalldauer), während Software-Builds drei- bis viermal schneller sein sollten.
Wenn Sie die perfekte DevOps-Kultur richtig einrichten, können die Ergebnisse erstaunlich sein. Wir haben festgestellt, dass ein Unternehmen eine DevOps-Struktur einführt und die Veröffentlichungsfenster durch Wiederholbarkeit und Skalierbarkeit von zwei Tagen auf zwei Stunden reduziert.
Das Ziel: Barrieren abbauen
Obwohl DevOps zunehmend an Bedeutung gewinnen, bleiben Fragen von denen, die über ihre Effektivität skeptisch sind. Es erinnert mich an eine Zeile, die ich oft auf Konferenzen und Shows höre, die immer wieder zum Lachen kommt: "Irren ist menschlich. In 10 Minuten auf 10.000 Servern zu irren ist DevOps. "
Die Wahrheit ist jedoch, dass DevOps ein sehr einfaches Konzept für Zusammenarbeit und intelligenteres Arbeiten ist. Es ist die Realisierung eines effizienteren, produktiveren Organisationsmodells. Es ist an der Zeit für alle Unternehmen, Barrieren abzubauen, eine moderne Kultur zu schaffen und bessere Software in kürzeren Zyklen zu liefern \cite{DevOps.2016}.
