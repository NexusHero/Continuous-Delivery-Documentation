\section{Diskussion}
Continuous Delivery ist ohne Frage ein lohnenswerter Schritt, den ein Unternehmen gehen kann, um seine Softwarequalität weiter zu verbessern und das Risiko beim Release drastisch zu minimieren. Wie bereits erwähnt, kann Continuous Delivery bzw. ein Teil davon auch ohne DevOps umgesetzt werden, allerdings gehen wir in diesem Artikel davon aus, dass Continuous Delivery vollständig eingeführt werden soll um alle Vorteile dieser Disziplin zu erhalten. Dabei reicht es nicht einfach nur …. Es gilt diverse Hürden zu überwinden, wie aus den obigen Kapiteln bereits hervorgegangen ist. Um eine geeignete DevOps-Kultur in das Unternehmen einzuführen bedeutet zunächst auch einen enormen initialen Aufwand für das Unternehmen. Auch im laufenden System fallen nun regelmäßig Weiterentwicklungen und womöglich Wartungen an.\\ \\
Die Erstellung von Tests muss zudem ein integraler Bestandteil der Entwicklung werden. Dabei muss entsprechendes Know-how zum Schreiben der Tests in den Teams verteilt und Testframeworks evaluiert werden. Die Tests müssen dabei unabhängig voneinander sein. Anschließend dreht sich sehr viel um die Automatisierung. Diese umfasst die Ausführung genannter Tests und die Bereitstellung resultierender Daten. Damit diese Tests zuverlässige Ergebnisse liefern, muss die Entwicklung und der Betrieb das Deployment und die Konfiguration der Anwendung gemeinsam planen, anpassen und warten [2].\\ \\
Dabei sollte dieser Ansatz möglichst zu Beginn eines Projekts verfolgt werden, um die Pipeline schrittweise aufzubauen. An eine bestehende Codebasis anzusetzen bedeutet oftmals eine fundamentale Änderung der bisherigen Techniken und Abläufe, denn wie bereits aus den obigen Kapiteln hervorgegangen ist, bedeutet das Einrichten einer DevOps-Kultur unter anderem eine Änderung an der gesamten Architektur des Systems. Dies bedeutet oftmals einen gewissen Leistungsdruck. Die Implementierung der Automatisierung benötigt ebenfalls Zeit. Der hier erbrachte Aufwand ist in dieser Zeit nicht für eine Weiterentwicklung des Systems bzw. der Features verfügbar.\\ \\
Eine enge Zusammenarbeit der Teams rückt in den Vordergrund. Dies lässt sich allerdings oft auch recht einfach gestalten. Dabei reicht es, die Arbeitsplätze der Mitarbeiter aus dem Betrieb direkt mit den Arbeitsplätzen der Entwicklung in einen Raum zusammenzulegen. So können beide Abteilungen viel einfacher miteinander kommunizieren, was die Zusammenarbeit wesentlich einfacher gestaltet. Dazu ist keine Änderung in der Organisation notwendig. DevOps kann so erreicht werden, indem man lediglich ein anderes Verständnis von Entwicklung und Betrieb bekommt, wodurch eine Änderung nicht unbedingt notwendig ist, allerdings sehr förderlich.\\ \\
DevOps beeinflusst also weit über Continuous Delivery hinaus die Prozesse, Teams und das Vorgehen. Gleichzeitig werden mit DevOps noch viele weitere Vorteile realisiert. Für eine vollständige Continuous-Delivery-Pipeline sind sowohl Fähigkeiten aus dem Betriebs- wie aus dem Entwicklungsumfeld notwendig. 


Ohne eine gewisse Kooperation kann eine Continuous-Delivery-Pipeline also nicht vollständig umgesetzt werden.!!!



Letztendlich lohnt es sich also, von Anfang an auf Continuous Delivery zu setzen, um so schrittweise die Continuous-Delivery-Pipeline aufzubauen und zu optimieren. Außerdem können so von Anfang an Technologie- und Architekturentscheidungen an Continuous Delivery ausgerichtet werden. So wird der Aufbau der Pipeline einfacher. Das bedeutet aber auch, dass Continuous Delivery bei neuen Projekten besonders einfach einführbar ist.


Wie in den vorhergehenden Abschnitten dargestellt, ist Continuous Delivery zusammen mit DevOps besonders sinnvoll. Schließlich benötigt die Continuous-Delivery-Pipeline in den verschiedenen Phasen Skills aus der Entwicklung (Dev) und dem Betrieb (Ops). Auf der anderen Seite ist De-vOps eine umfassende Änderung in der Organisation. Gerade in großen Unternehmen sind die Bereiche Entwicklung und Betrieb oft direkt unter dem CIO schon getrennt. Tatsächlich DevOps einzuführen, würde diese Organisation fundamental ändern, weil dann Teams Entwickler und Betriebler umfassen würden. Eine solche fundamentale Änderung der Organisation ist aber aufwendig und verursacht erhebliche Widerstände.

Daher stellt sich die Frage, ob Continuous Delivery auch ohne DevOps eingeführt werden kann. Für Continuous Delivery muss es nicht unbedingt gemischte Teams geben, sondern kollaborative Arbeit an der Continuous-Delivery-Pipeline ist notwendig. Konkret sollte der Betrieb an der Automatisierung des Deployments arbeiten und die Entwicklung vor allem an den verschiedenen Testphasen. Wechselseitige Unterstützung darüber hinaus ist natürlich auch denkbar und sinnvoll. Für eine solche gemeinsame Arbeit an der Pipeline muss es aber nicht zwingend eine Umorganisation geben.

Organisatorische Grenzen schlagen sich aber oft auch in der Pipeline nieder: Beispielsweise kann die Entwicklung Testsysteme aufbauen und dafür eine Automatisierung etablieren. Wenn der Betrieb aber die Automatisierung nicht übernehmen will, sondern seine eigene Automatisierung aufbauen will, zerbricht die Pipeline in zwei Teile. Die Systeme und die Automatisierung in der Entwicklung werden anders aufgebaut sein als diejenigen im Betrieb. Das macht es schwierig, Probleme in Produktion auf den Systemen der Entwicklung nachzustellen, und verursacht doppelte Aufwände, weil Änderungen in beiden Systemen nachvollzogen werden müssen. Das gilt selbst dann, wenn die Entwicklung und der Betrieb dieselben Werkzeuge nutzen und der Betrieb die Änderungen der Entwicklung manuell übernehmen will. Die beiden Automatisierungsansätze werden sich auseinanderentwickeln und sind später nur mit erheblich Aufwand wieder zu vereinheitlichen. Diesem technischen Problem liegt ein organisatorisches zu Grunde: Der Betrieb vertraut Änderungen der Entwicklung nicht und will sie nicht übernehmen. Dieses Problem ist allerdings ohne Umstellung der Organisation lösbar – eben durch bessere Kollaboration. Wenn dieses Misstrauen beseitigt ist, wird sich auch eine technische Lösung ergeben. Umgekehrt wird keine technische Lösung dieses Problem beseitigen können.

Continuous Delivery Pipeline abbrechen
Wenn es gar keine Unterstützung durch den Betrieb gibt, stellt sich die Frage, ob Continuous Delivery überhaupt umsetzbar ist. Schließlich ist ja ein Ziel von Continuous Delivery, Software schneller in Produktion zu bringen. Das ist wohl kaum möglich, wenn der Betrieb die Continuous-Delivery-Bestrebungen nicht unterstützt und die Pipeline daher nicht in die Produktion gehen kann. Es wäre aber dennoch möglich, die Continuous-Delivery-Pipeline mit den verschiedenen Testphasen umzusetzen und vor der Produktion abzubrechen. Dann erreicht man aber kein schnelleres Deployment in Produktion.

Aber Time-to-Market ist nicht das einzige Ziel von Continuous Delivery. Ein weiteres Ziel ist die Reproduzierbarkeit und die Wiederholbarkeit. Die wird durch eine solche verkürzte Continuous-Delivery-Pipeline immer noch erreicht. Alle Tests sind jederzeit reproduzierbar, sie werden bei jeder Änderung wiederholt und die Tests werden viel öfter ausgeführt. Das führt zu einer höheren Softwarequalität und zu schnellerem Feedback – einem wesentlichen Ziel von Continuous Delivery. Ebenso muss die Software mindestens auf Testsystemen installiert werden, so dass auch die Installation prinzipiell reproduzierbar ist.

Die höhere Softwarequalität und das automatisierte und damit reproduzierte Deployen der Anwendung zumindest auf Testsysteme ist schon Grund genug für die Einführung von Continuous Delivery. Es ist also nicht nur möglich, Continuous Delivery ohne DevOps einzuführen, sondern eine reduzierte Version kann sogar ganz ohne Unterstützung des Betriebs umgesetzt werden.

Leider erreicht man so zwar nicht alle Vorteile von Continuous Delivery, aber es ist auf jeden Fall trotzdem ein sinnvoller Schritt. Außerdem kann es gut sein, dass der Betrieb später die Continuous-Delivery-Pipeline auch in Produktion fortsetzt.

Eine Continuous-Delivery-Pipeline, die nur durch den Betrieb umgesetzt wird, würde nur das automatisierte Ausrollen in Produktion umfassen. Automatisierung sollte eine Betriebsabteilung sowieso umsetzen, um Aufwände in Grenzen zu halten. Aber erst durch die Testphasen ist ein schnelles Deployment in Produktion wirklich realistisch, weil nur dann die Software nachweislich eine Qualität hat, wie sie für die Produktion notwendig ist. Selbst wenn das Deployment an sich vollständig automatisiert ist und zuverlässig funktioniert, wird niemand neue Software in Produktion bringen, weil erst die Tests die dazu notwendige Qualität absichern.

Es ist also denkbar, Continuous Delivery auch ohne DevOps umzusetzen, wenngleich diese Ansätze natürlich weniger Vorteile erbringen als eine Einführung mit DevOps zusammen.


DevOps ergänzt Continuous Delivery um ein organisatorisches Modell, bei dem Entwicklung und Betrieb enger kooperieren. Bei Continuous Delivery geht es um die Auslieferung von Software. Dazu müssen diese beiden Abteilungen eng zusammenarbeiten, so dass DevOps gerade für Continuous Delivery hilfreich ist. Aber die Kollaboration kann viel weiter gehen – es können auch Monitoring und Troubleshooting durch die bessere Kollaboration vereinfacht werden. Der nächste Schritt können noch breiter aufgestellte Teams sein, wie es Design Thinking [1] vorschlägt. Lean Startup [2] stellt das Bewerten von Features in den Mittelpunkt, um dadurch das Produkt schrittweise zu verbessern und zu verfeinern. Da mit Continuous Delivery die Auslieferung und Auswertung einzelner Features ohne Weiteres möglich sind, kann Lean Startup gerade mit Continuous Delivery gut umgesetzt werden.



% Can use something like this to put references on a page
% by themselves when using endfloat and the captionsoff option.
\ifCLASSOPTIONcaptionsoff
  \newpage
\fi