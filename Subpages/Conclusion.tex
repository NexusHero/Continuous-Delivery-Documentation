\section{Diskussion und Ausblick}
Continuous Delivery ist ohne Frage ein lohnenswerter Schritt, den ein Unternehmen gehen kann, um seine Softwarequalität weiter zu verbessern und das Risiko beim Release drastisch zu minimieren. Wie bereits erwähnt, kann Continuous Delivery bzw. ein Teil davon auch ohne DevOps umgesetzt werden. Da es teilweise schon ausreicht, eine gewisse Zusammenarbeit zu ermöglichen, müssen die Teams nicht vermischt bzw. die Organisation umstrukturiert werden. Diese Zusammenarbeit erreicht man bereits, wenn die Arbeitsplätze der benötigten Abteilungen zusammengelegt werden. Diese erleichtert die Kommunikation der Mitarbeiter wesentlich. Dies ist bereits ein sinnvoller Schritt für Unternehmen, es kann jedoch nicht der ganze Nutzen aus Continuous Delivery gezogen werden. Zu Beginn des Artikels wurde allerdings betont, dass Continuous Delivery vollständig eingeführt werden soll um alle Vorteile dieser Disziplin zu erhalten. Dabei reicht es nicht, entsprechende Werkzeuge anzuwenden und die Kommunikation der Teams anzupassen. Es gilt diverse Hürden zu überwinden, wie aus den obigen Kapiteln bereits hervorgegangen ist. \\ \\
Continuous Delivery bedeutet eine Einführung von Microservices. Hier werden Anforderungen wie Unabhängigkeit, Technologiefreiheit etc. sogar priorisiert. Jeder Service soll schließlich von einem Team entwickelt und betrieben werden. Diese Dienste bilden eine geschlossene Fachlichkeit ab. Das heißt, dass es auch keine technisch orientierten Teams mehr gibt. Jedes Team braucht Know-How aus verschiedenen Bereichen, um die Dienste zu implementieren. Alleine dadurch zeigt sich, dass es eine Umorganisation des Unternehmens erfordert. Eine geeignete DevOps-Kultur einzuführen bedeutet zunächst einen enormen initialen Aufwand für das Unternehmen. Soll das bestehende System mit diesem Architekturansatz umgesetzt werden, bedeutet dies eine Modularisierung seiner Software. Dadurch, dass es bei Microservice-basierten Systemen zu tausenden Deployments pro Jahr kommen kann, bietet die Delivery-Pipeline mit ihrer Automatisierung eine notwendige Technik, um anfallende Aufwände im Zaum zu halten. Auch im laufenden System fallen regelmäßig Weiterentwicklungen und womöglich Wartungen dieser Dienste an. Die Erstellung von Tests muss zudem ein integraler Bestandteil der Entwicklung werden. Dabei muss entsprechendes Know-how zum Schreiben der Tests in den Teams verteilt und Testframeworks evaluiert werden. Die Tests müssen dabei unabhängig voneinander sein. Anschließend dreht sich viel um die Automatisierung. Diese umfasst die Ausführung genannter Tests und die Bereitstellung resultierender Daten. Damit diese Tests zuverlässige Ergebnisse liefern, muss die Entwicklung und der Betrieb das Deployment und die Konfiguration der Anwendung gemeinsam planen, anpassen und warten \cite{Birk.2014} \\ \\
Dabei sollte dieser Ansatz möglichst zu Beginn eines Projekts verfolgt werden, um die Pipeline schrittweise aufzubauen und Entscheidungen bezüglich Technologie und Architektur darauf auszurichten. An eine bestehende Codebasis anzusetzen bedeutet oftmals eine fundamentale Änderung der bisherigen Techniken und Abläufe, denn wie bereits aus den obigen Kapiteln hervorgegangen ist, bedeutet das Einrichten einer DevOps-Kultur unter anderem eine Änderung an der gesamten Architektur des Systems. Die Implementierung der Automatisierung benötigt ebenfalls Zeit. Der hier erbrachte Aufwand ist in dieser Zeit nicht für eine Weiterentwicklung des Systems bzw. der Features verfügbar. Continuous Delivery beeinflusst somit weitgehend Elemente eines Unternehmens und ohne die Ergänzung des organisatorischen Modells DevOps, kann dieser Ansatz nicht vollständig umgesetzt werden.\\ \\
Da Microservices aktuell einen beliebten Architekturansatz darstellen, nehmen Continuous Delivery und Continuous Deployment wohl eine immer wichtigere Rolle in der Softwareentwicklung ein. Allerdings gibt es bezüglich dieser Disziplin in Verbindung mit DevOps nur einseitige Berichte. Es wird meist nur beschrieben, dass DevOps der Schlüssel sei. Offen bleibt die Frage, wie gut die Zusammenarbeit von Betrieb und Entwicklung tatsächlich funktioniert. Auch wenn ein gemeinsames Arbeiten stattfindet, haben sie dennoch unterschiedliche Perspektiven. Wie kann die Kultur geändert werden, wenn sich Mitarbeiter über Jahre auf die vorherrschende Struktur im Unternehmen festgefahren haben? 


% Can use something like this to put references on a page
% by themselves when using endfloat and the captionsoff option.
\ifCLASSOPTIONcaptionsoff
  \newpage
\fi