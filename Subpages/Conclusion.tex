\section{Diskussion}
Continuous Delivery bzw. eine reduzierte Form der Pipeline kann ohne DevOps umgesetzt werden. Da es teilweise schon ausreicht, eine gewisse Zusammenarbeit der Abteilungen zu ermöglichen, müssen die Teams nicht vermischt bzw. die Organisation umstrukturiert werden. Diese Zusammenarbeit erreicht man bereits, wenn relevante Arbeitsplätze zusammengelegt werden, was die Kommunikation der Mitarbeiter wesentlich erleichtert. Dies ist bereits ein sinnvoller Schritt für Unternehmen, es kann jedoch nicht der ganze Nutzen aus Continuous Delivery gezogen werden. Zu Beginn des Artikels wurde betont, dass Continuous Delivery vollständig eingeführt werden soll, um alle Vorteile dieser Disziplin zu erhalten. Dabei reicht es nicht, die Kommunikation der Teams anzupassen. Die Verwendung entsprechender Werkzeuge und Produkte wird keinen organisatorischen Wandel herbeiführen. Continuous Delivery beeinflusst weitgehend Elemente eines Unternehmens und ohne die Ergänzung eines organisatorischen Modells, kann dieser Ansatz nicht vollständig umgesetzt werden.\\ \\
Continuous Delivery bedeutet eine Einführung von Microservices. Hier werden Anforderungen wie Unabhängigkeit, Technologiefreiheit sogar priorisiert. Jeder Service soll schließlich von einem Team entwickelt und betrieben werden. Diese Dienste bilden eine geschlossene Fachlichkeit ab. Das heißt, dass es keine technisch orientierten Teams mehr gibt. Jedes Team braucht Know-How aus verschiedenen Bereichen, um die Dienste zu implementieren. Alleine dadurch zeigt sich, dass es eine Umorganisation des Unternehmens erfordert. Eine solche Kultur zu etablieren stellt zunächst einen enormen initialen Aufwand für das Unternehmen dar. Soll das bestehende System mit diesem Architekturansatz umgesetzt werden, bedeutet dies eine Modularisierung seiner Software. Die Erstellung voneinander unabhängiger Tests und das Automatisieren der Entwicklungsprozesse müssen integrale Bestandteile der Entwicklung werden. Dabei sollte dieser Ansatz möglichst zu Beginn eines Projekts verfolgt werden, um die Pipeline schrittweise aufzubauen und Entscheidungen bezüglich Technologie und Architektur darauf auszurichten. Die Implementierung der Automatisierung benötigt ebenfalls Zeit. Der hier erbrachte Aufwand ist in dieser Zeit nicht für eine Weiterentwicklung des Systems verfügbar. \\ \\
Die hier aufgeführten Hürden wurden alle in der Arbeit von (...) behandelt und über Case Studies belegt. Mit diesem Wissen, eine strukturierten Planung und einer kompetenten Führungsebene, welche selten von der Transformation betroffen ist, kann eine Organisation hier eine Reihe von Optimierungen erzielen. Continuous Delivery ist ohne Frage ein lohnenswerter Schritt, den ein Unternehmen gehen kann, um die Softwarequalität zu verbessern und das Risiko beim Release drastisch zu minimieren. Der größte Vorteil, der hier erwartet werden kann liegt vermutlich auch in der beschleunigten Markteinführungszeit seiner Releases. 

\section{Ausblick}
Da Microservices aktuell einen beliebten Architekturansatz darstellen, nehmen Continuous Delivery und Continuous Deployment wohl eine immer wichtigere Rolle in der Softwareentwicklung ein. Allerdings gibt es bezüglich dieser Disziplin in Verbindung mit DevOps nur einseitige Berichte. Es wird meist nur beschrieben, dass DevOps der Schlüssel sei. Offen bleibt die Frage, wie gut die Zusammenarbeit von Betrieb und Entwicklung tatsächlich funktioniert. Auch wenn ein gemeinsames Arbeiten stattfindet, haben die Abteilungen dennoch unterschiedliche Perspektiven. Wie kann die Kultur geändert werden, wenn sich Mitarbeiter über Jahre auf die vorherrschende Struktur im Unternehmen festgefahren haben? An dieser Forschungsfrage können weitere Arbeiten anknüpfen. 


% Can use something like this to put references on a page
% by themselves when using endfloat and the captionsoff option.
\ifCLASSOPTIONcaptionsoff
  \newpage
\fi