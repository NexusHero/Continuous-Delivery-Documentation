\section{Diskussion}
Continuous Delivery ist ohne Frage ein lohnenswerter Schritt, den ein Unternehmen gehen kann, um seine Softwarequalität weiter zu verbessern und das Risiko beim Release drastisch zu minimieren. Der Code kann dabei stets auf einem konstanten Niveau gehalten werden. Durch die kleinen Features wird innerhalb sehr kurzer Zeit wertvolles Feedback gewonnen, welches sich auch für Experimente eignet. Zeigt der Kunde Interesse am neuen Feature? Lohnt es sich, hier weiter zu investieren? Es lässt sich die Durchlaufzeiten von der Idee bis zur Produktion auf einen sehr geringen Zeitraum reduzieren, wodurch es das Geld erheblich früher als bei großen Deployment Monolithen einbringt. Zudem bietet dieser Ansatz großes Sparpotenzial, da in Produktion entdeckte Fehler häufig mit hohen Kosten verbunden sind. Ein weiterer großer Vorteil ist die schnelle Reaktion auf sich ändernde Anforderungen des Kunden oder Änderungen am Markt. Allerdings muss ein Unternehmen auch einiges aufwenden, wie in der Arbeit bereits beschrieben wurde. 
Es reicht nicht einfach nur …. Es gilt diverse Hürden zu überwinden, um eine geeignete DevOps-Kultur in das Unternehmen einzuführen. Eine Kultur ist dabei nicht greifbar und schwer zu ändern. Diese Entscheidungen bringen dafür ein völlig anderes Arbeitsverhältnis hervor …. Eine enge Zusammenarbeit rückt in den Vordergrund und erlaubt es…



\appendices
\section{Proof of the First Zonklar Equation}
Appendix one text goes here.

% you can choose not to have a title for an appendix
% if you want by leaving the argument blank
\section{}
Appendix two text goes here.


% Can use something like this to put references on a page
% by themselves when using endfloat and the captionsoff option.
\ifCLASSOPTIONcaptionsoff
  \newpage
\fi