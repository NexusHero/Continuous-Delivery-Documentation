\section{Diskussion}
Continuous Delivery ist ohne Frage ein lohnenswerter Schritt, den ein Unternehmen gehen kann, um seine Softwarequalität weiter zu verbessern und das Risiko beim Release drastisch zu minimieren. Wie bereits erwähnt, kann Continuous Delivery bzw. ein Teil davon auch ohne DevOps umgesetzt werden. Da es teilweise schon ausreicht, eine gewisse Zusammenarbeit zu ermöglichen, müssen die Teams nicht vermischt bzw. die Organisation umstrukturiert werden. Diese Zusammenarbeit erreicht man bereits, wenn die Arbeitsplätze der beiden Abteilungen zusammengelegt werden. Diese erleichtert die Kommunikation der Mitarbeiter wesentlich. Dies ist bereits ein sinnvoller Schritt für Unternehmen, es kann allerdings nicht der ganze Nutzen aus Continuous Delivery gezogen werden. Zu Beginn des Artikels wurde davon ausgegangen, dass Continuous Delivery vollständig eingeführt werden soll um alle Vorteile dieser Disziplin zu erhalten. Dabei reicht es nicht, entsprechende Werkzeuge anzuwenden und die Kommunikation der Teams anzupassen. Es gilt diverse Hürden zu überwinden, wie aus den obigen Kapiteln bereits hervorgegangen ist. \\ \\
Um eine geeignete DevOps-Kultur in das Unternehmen einzuführen bedeutet zunächst auch einen enormen initialen Aufwand für das Unternehmen. Auch im laufenden System fallen nun regelmäßig Weiterentwicklungen und womöglich Wartungen an. Die Erstellung von Tests muss zudem ein integraler Bestandteil der Entwicklung werden. Dabei muss entsprechendes Know-how zum Schreiben der Tests in den Teams verteilt und Testframeworks evaluiert werden. Die Tests müssen dabei unabhängig voneinander sein. Anschließend dreht sich sehr viel um die Automatisierung. Diese umfasst die Ausführung genannter Tests und die Bereitstellung resultierender Daten. Damit diese Tests zuverlässige Ergebnisse liefern, muss die Entwicklung und der Betrieb das Deployment und die Konfiguration der Anwendung gemeinsam planen, anpassen und warten [2].\\ \\
Dabei sollte dieser Ansatz möglichst zu Beginn eines Projekts verfolgt werden, um die Pipeline schrittweise aufzubauen und Entscheidungen bezüglich Technologie und Architektur darauf auszurichten. An eine bestehende Codebasis anzusetzen bedeutet oftmals eine fundamentale Änderung der bisherigen Techniken und Abläufe, denn wie bereits aus den obigen Kapiteln hervorgegangen ist, bedeutet das Einrichten einer DevOps-Kultur unter anderem eine Änderung an der gesamten Architektur des Systems. Dies bedeutet oftmals einen gewissen Leistungsdruck. Die Implementierung der Automatisierung benötigt ebenfalls Zeit. Der hier erbrachte Aufwand ist in dieser Zeit nicht für eine Weiterentwicklung des Systems bzw. der Features verfügbar.\\ \\
Continuous Delivery beeinfluss somit weitgehend Elemente eines Unternehmens und ohne die Ergänzung des organisatorischen Modells DevOps, kann dieser Ansatz nicht vollständig umgesetzt werden. Die Umsetzung bedeutet hierbei eine Einführung von Microservices, welche entsprechende Anforderungen sogar priorisieren. Dadurch, dass es bei Microservice-basierten Systemen zu tausenden Deployments pro Jahr kommen kann, bietet die Delivery-Pipeline mit ihrer Automatisierung eine notwendige Technik, um anfallende Aufwände im Zaum zu halten.


% Can use something like this to put references on a page
% by themselves when using endfloat and the captionsoff option.
\ifCLASSOPTIONcaptionsoff
  \newpage
\fi