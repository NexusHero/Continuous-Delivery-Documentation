\section{Diskussion}
Continuous Delivery ist ohne Frage ein lohnenswerter Schritt, den ein Unternehmen gehen kann, um seine Softwarequalität weiter zu verbessern und das Risiko beim Release drastisch zu minimieren. Wie bereits erwähnt, kann Continuous Delivery bzw. ein Teil davon auch ohne DevOps umgesetzt werden, allerdings gehen wir in diesem Artikel davon aus, dass Continuous Delivery vollständig eingeführt werden soll um alle Vorteile dieser Disziplin zu erhalten. Dabei reicht es nicht einfach nur …. Es gilt diverse Hürden zu überwinden, wie aus den obigen Kapiteln bereits hervorgegangen ist. Um eine geeignete DevOps-Kultur in das Unternehmen einzuführen bedeutet zunächst auch einen enormen initialen Aufwand für das Unternehmen. Auch im laufenden System fallen nun regelmäßig Weiterentwicklungen und womöglich Wartungen an.\\ \\
Die Erstellung von Tests muss zudem ein integraler Bestandteil der Entwicklung werden. Dabei muss entsprechendes Know-how zum Schreiben der Tests in den Teams verteilt und Testframeworks evaluiert werden. Die Tests müssen dabei unabhängig voneinander sein. Anschließend dreht sich sehr viel um die Automatisierung. Diese umfasst die Ausführung genannter Tests und die Bereitstellung resultierender Daten. Damit diese Tests zuverlässige Ergebnisse liefern, muss die Entwicklung und der Betrieb das Deployment und die Konfiguration der Anwendung gemeinsam planen, anpassen und warten [2].\\ \\
Dabei sollte dieser Ansatz möglichst zu Beginn eines Projekts verfolgt werden, um die Pipeline schrittweise aufzubauen. An eine bestehende Codebasis anzusetzen bedeutet oftmals eine fundamentale Änderung der bisherigen Techniken und Abläufe, denn wie bereits aus den obigen Kapiteln hervorgegangen ist, bedeutet das Einrichten einer DevOps-Kultur unter anderem eine Änderung an der gesamten Architektur des Systems. Dies bedeutet oftmals einen gewissen Leistungsdruck. Die Implementierung der Automatisierung benötigt ebenfalls Zeit. Der hier erbrachte Aufwand ist in dieser Zeit nicht für eine Weiterentwicklung des Systems bzw. der Features verfügbar.\\ \\
Eine enge Zusammenarbeit der Teams rückt in den Vordergrund. Dies lässt sich allerdings oft auch recht einfach gestalten. Dabei reicht es, die Arbeitsplätze der Mitarbeiter aus dem Betrieb direkt mit den Arbeitsplätzen der Entwicklung in einen Raum zusammenzulegen. So können beide Abteilungen viel einfacher miteinander kommunizieren, was die Zusammenarbeit wesentlich einfacher gestaltet. Dazu ist keine Änderung in der Organisation notwendig. DevOps kann so erreicht werden, indem man lediglich ein anderes Verständnis von Entwicklung und Betrieb bekommt, wodurch eine Änderung nicht unbedingt notwendig ist, allerdings sehr förderlich.\\ \\
DevOps beeinflusst also weit über Continuous Delivery hinaus die Prozesse, Teams und das Vorgehen. Gleichzeitig werden mit DevOps noch viele weitere Vorteile realisiert. Für eine vollständige Continuous-Delivery-Pipeline sind sowohl Fähigkeiten aus dem Betriebs- wie aus dem Entwicklungsumfeld notwendig. Ohne eine gewisse Kooperation kann eine Continuous-Delivery-Pipeline also nicht vollständig umgesetzt werden.

% Can use something like this to put references on a page
% by themselves when using endfloat and the captionsoff option.
\ifCLASSOPTIONcaptionsoff
  \newpage
\fi