\IEEEraisesectionheading{\section{Introduction}\label{sec:introduction}}
\IEEEPARstart{M}{it} einem DevOps-Team können beide Bereiche abgedeckt und so die Software weiterentwickelt und betrieben werden. Außerdem kann das Feedback aus der Produktion direkt in die Weiterentwicklung einfließen, weil das Team alle dazu notwendigen Rollen vereinigt. So kann das Ziel von Continuous Delivery, nämlich schnelles Feedback, erreicht werden. Oft wird Continuous Delivery sogar mit DevOps gleichgesetzt. Das trifft aber nicht den Kern der Sache. Continuous Delivery wird zwar durch DevOps deutlich vereinfacht und ist auch eine wesentliche Praktik im DevOps-Umfeld. Aber neben Continuous Delivery gibt es noch viel mehr Bereiche, in denen DevOps als Organisationsform hilfreich ist: Wo möchte ein Unternehmen hin? Zielsetzung. \\In diesem wissenschaftlichen Artikel wird ein möglicher Pfad zum Ziel beleuchtet. Dabei gibt es unterschiedliche Voraussetungen für Unternehmen. Beispielsweise ist es vom aktuellen Zustand der Organisation bzw. wann sie sich für diesen Schritt entscheidet abhängig. Ist das Projekt bereits fortgeschritten und muss die Bestehende Pipeline entsprechend erweitert werden oder kann das Projekt direkt von Anfang an darauf ausgelegt werden.
Es verspricht eine Bereitstellung seiner Software auf Knopfdruck und das bei einer sehr guten Qualität. Erreicht wird dies durch eine radikale Automatisierung. Schritte wie Entwicklung, Qualitätssicherung und Auslieferung werden in kleinere Einheiten abgebildet. Bei Änderungen jeglicher Art, seien es neue Funktionen, Fehlerbehebungen oder nur Konfigurationsänderungen, wird die Software direkt im Anschluss automatisiert getestet. Durch das daraus resultierende schnelle Feedback können Seiteneffekte und Regressionen direkt erkannt werden. Die Automatisierung erfolgt durch eine Pipeline \cite{Birk.2014}. 

