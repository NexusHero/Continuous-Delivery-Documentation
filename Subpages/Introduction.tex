\IEEEraisesectionheading{\section{Introduction}\label{sec:introduction}}
\IEEEPARstart{D}{urch} Continuous Integration hat man bereits einen stabilen Prozess, um qualitativ hochwertige Software in Produktion zu bringen. Die Prinzipien, die bei Continuous Delivery verfolgt werden klingen im ersten Moment vielversprechend. Was muss ein Unternehmen aufwenden, um diesen Schritt zu gehen und welche Auswirkungen hat diese Entscheidung auf das Unternehmen? 
Continuous Delivery ermöglicht es, Software schneller und mit wesentlich höherer Zuverlässigkeit in Produktion zu bringen. Grundlage dafür ist eine Continuous-Delivery-Pipeline, die das Ausrollen der Software weitgehend automatisiert und so einen reproduzierbaren, risikoarmen Prozess für die Bereitstellung neuer Releases darstellt [1]. Es verspricht eine Bereitstellung seiner Software auf Knopfdruck und das bei einer sehr guten Qualität. Erreicht wird dies durch eine radikale Automatisierung. Schritte wie Entwicklung, Qualitätssicherung und Auslieferung werden in kleinere Einheiten abgebildet. Bei Änderungen jeglicher Art, seien es neue Funktionen, Fehlerbehebungen oder nur Konfigurationsänderungen, wird die Software direkt im Anschluss automatisiert getestet. Durch das daraus resultierende schnelle Feedback können Seiteneffekte und Regressionen direkt erkannt werden. Die Automatisierung erfolgt durch eine Pipeline [2]. 

\subsection{Kosten}
Kosten, Lizenzen, Mitarbeiterschulung

Zielsetzung: wo möchte ein Unternehmen hin?

Unterschiedliche Voraussetzungen für Unternehmen bzg. continuous Delivery. Nihct jeder ist gleich.


Ausblick fehlt --> Wo sieht man noch Möglichkeiten? Werkzeugunterstützung?

Präsentation: 

Schwerpunkte setzten: Zieldefinition: Wie kommt ein Unternehmen da hin.

wie mache ich das? Projektweise? Unternehmensweise? 
Widerstände im Unternehmen.

