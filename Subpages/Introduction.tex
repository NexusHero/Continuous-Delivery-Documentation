\IEEEraisesectionheading{\section{Einführung}\label{sec:introduction}}
\IEEEPARstart{C}ontinuous Integration wird bereits seit langem von vielen Unternehmen genutzt \cite{automic.2017}. Dadurch haben sie bereits einen stabilen Prozess, um nachhaltig qualitativ hochwertige Software in Produktion zu bringen. An dieser Stelle setzt Continuous Delivery an. Die Prinzipien, welche bei Continuous Delivery verfolgt werden, klingen im ersten Moment vielversprechend, denn es ermöglicht das Ausrollen von Software mit einer wesentlich höheren Geschwindigkeit sowie Zuverlässigkeit. Auch wiederkehrende Herausforderungen wie Seiteneffekten und Regressionen, lassen sich mit diesem Ansatz vorbeugen. Obwohl Continuous Delivery lediglich eine Sammlung von Techniken, Prozessen und Werkzeugen ist, stellt das Aufsetzen dieser Disziplin vor allem in großen Unternehmen mit unflexiblen Strukturen eine Herausforderung dar \cite{Wolff.2016}. 
\begin{quote} \textit{\glqq Organizations which design systems […] are constrained to produce designs which are copies of the communication structures of these organizations. \grqq~}\cite[S.5]{Farley.2011} \end{quote} 

Laut dem Gesetz von Conway bedeutet Continuous Delivery einen Bruch mit der bisherigen Organisation im Unternehmen. Da sich Continuous Delivery um das Ausliefern von Software dreht, müssen Entwicklung und Betrieb eng miteinander kooperieren. Es bedarf somit einer Ergänzung um ein entsprechendes organisatorisches Modell - DevOps. In dieser Arbeit wird ein Pfad beleuchtet, welchen ein Unternehmen gehen muss, um eine vollständige Continuous-Delivery-Pipeline aufzusetzen. Dabei gibt es unterschiedliche Voraussetzungen für Unternehmen. Beispielsweise hängt es vom aktuellen Zustand der Organisation bzw. zu welchem Zeitpunkt sie sich für diesen Schritt entscheidet ab. Ist das Projekt bereits fortgeschritten und muss die Bestehende Pipeline entsprechend erweitert werden oder kann das Projekt direkt von Anfang an darauf ausgelegt werden?
\\ \\ Zu Beginn der Arbeit erfolgt somit eine kurze Erläuterung von DevOps, die Prämisse für die unbeschränkte Umsetzung von Continuous Delivery. Darauf aufbauend wird beschrieben, welche Teile einer Organisation bei der Entwicklung zur DevOps-Kultur betroffen sind welche Barrieren es hier zu bewältigen gibt. Es wird untersucht, was ein Unternehmen aufwenden muss, um diesen Schritt zu gehen und welche Auswirkungen diese Entscheidung auf das Unternehmen hat. Abschließend wird in der Diskussion erläutert, welche Änderungen der Unternehmenskultur zu erwarten sind, wenn alle Vorteile von Continuous Delivery nutzen werden sollen.